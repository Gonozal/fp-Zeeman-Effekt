% Klassifiziert den Dokumenten-Typ
% Doku: http://exp1.fkp.physik.tu-darmstadt.de/tuddesign/
% Farben: http://www.tu-darmstadt.de/media/medien_stabsstelle_km/services/medien_cd/das_bild_der_tu_darmstadt.pdf
%  bigchapter: Chapter haben doppelte Schriftgröße
%  linedtoc: Linien im Inhaltsverzeichnis wie bei Überschriften
%  colorbacktitle: Der Dokumenten-Titel wird mir der Accentfarbe hinterlegt
\documentclass[bigchapter,colorback,accentcolor=tud4b,linedtoc,11pt]{tudreport}

% Input Dokument hat das Encoding UTF-8
\usepackage[utf8]{inputenc}
% Wichtiges Paket für Links und verlinktes Inhaltsverzeichnis
\usepackage[ngerman]{hyperref}
% Paket für Fußnoten
\usepackage[stable]{footmisc}
% Paket für amsmath (aligned mathe formeln)
\usepackage{amsmath}
% Paket für Bibliotheks-Verzeichnis, square: Verwende eckige statt runde klammern
% \usepackage[square]{natbib}
% Paket zum Plotten von Datensätzen
\usepackage{pgfplots}
\pgfkeys{%
  /pgfplots/default/.style={%
    /pgf/number format/use comma,
    legend pos=north east,
    x tick label style={/pgf/number format/1000 sep=},
    y tick label style={/pgf/number format/1000 sep=},
    width=0.9\linewidth,
    height=0.40\linewidth,
    scale only axis,
    grid=both,
    tick align=outside,
    tickpos=left,
    minor x tick num=3,
    minor y tick num=4,
    minor grid style={dotted,thin}
  }
}

% Anhänge für Original-Messdaten
\usepackage{fancyvrb}

% Verwende deutsche Bezeichner für Inhaltsverzeichnis, ... (ngerman = New German: neue Rechtschreibung)
\usepackage{ngerman}
% Deutsche Zahlen (entfernt z.B. das Leerzeichen nach einem Dezimal-Komma)
\usepackage{ziffer} 

\usepackage[verbose]{placeins}

%wegen Grafikverschiebung hinzugefügt
\usepackage{float}

%\usepackage{graphicx}
%\usepackage{caption}
\usepackage{subcaption} %Für subfigures

% PDF-Optionen
\hypersetup{%
  pdftitle={TU Darmstadt \- Physikalisches Praktikum für Fortgeschrittene},
  pdfauthor={Esra Bauer, Sören Link und Christian Hab},
  pdfsubject={Versuch 1.5},
  pdfview=FitH,
}
% Nummeriere formeln in Subsections einzeln
% Kleines makro zur assymetrischen Fehlerangabe

% Entspricht-Zeichen
\usepackage{scalerel}

\newcommand\equalhat{%
\let\savearraystretch\arraystretch
\renewcommand\arraystretch{0.3}
\begin{array}{c}
\stretchto{
    \scalerel*[\widthof{=}]{\wedge}
    {\rule{1ex}{3ex}}%
}{0.5ex}\\ 
=%
\end{array}
\let\arraystretch\savearraystretch
}
%BEGINN TITELSEITE

\title{Zeeman-Effekt}

\subtitle{Esra Bauer  \\Sören Link \\Christian Hab}

\subsubtitle{Betreuer: Matthias Sattig \hfill Versuchsdatum: 13. Januar 2015}

\author{Esra Bauer, Sören Link, Christian Hab}

%\settitlepicture{img/title.jpg}

\institution{Physikalisches Praktikum \\für Fortgeschrittene \\ Versuch 1.5}

\date{\today}
%ENDE TITELSEITE


\begin{document}
%ANFANG DOKUMENT

%Titelseite einfügen
\maketitle

%Inhaltsverzeichnis einfügen
\tableofcontents

%ANFANG INHALT

\chapter{Einleitung}

\chapter{Grundlagen}
\section{Fabry-Pérot-Interferometer}

% \begin{figure}[h] 
%   \centering
%      \includegraphics[width=0.7\textwidth]{img/...}
%   \caption{...}
%   \label{fig:...}
% \end{figure}

\section{Klassische Erklärung des Zeeman-Effekts}
\section{Quantenmeschanische Kopplungen im Atom}
\subsection{LS-Kopplung}
\subsection{jj-Kopplung}
\section{Quantenmeschanische Erklärungen}
\subsection{Quantenmeschanische Erklärung des Zeeman-Effekts}
\subsection{Quantenmeschanische Erklärung des Paschen-Back-Effekts}
\section{Auswahlregeln für Dipolübergänge}

\chapter{Durchführung}
\section{Messung von $\frac{\delta\alpha_1}{\delta\alpha_2}$}
%\begin{center}
%  \begin{tabular}{|p{2.2cm}|p{4.5cm}|p{3cm}|p{4cm}|}
%    \hline
%    Messungs-Nummer & Kristallisationstemperatur der Probe in °C & Messmethode    & Name der Messdaten-Datei \\ \hline
%    1               & 140                                        & Winkelabhängig & PET-140.0.txt \\ \hline
%    2               & 140                                        & Wanderspalt    & PET-140.ms \\ \hline
%    3               & 170                                        & Wanderspalt    & PET-170.ms \\ \hline
%    4               & 170                                        & Winkelabhängig & PET-170.0.txt \\ \hline
%    5               & 170 (Selbst getempert)                     & Winkelabhängig & PET-170-eigen.0.txt \\ \hline
%    6               & 170 (Selbst getempert)                     & Wanderspalt    & PET-170-eigen.ms \\ \hline
%    7               & 200                                        & Wanderspalt    & PET-200.ms \\ \hline
%    8               & 200                                        & Winkelabhängig & PET-200.0.txt \\ \hline
%    9               & 215                                        & Winkelabhängig & PET-215.0.txt \\ \hline
%    10              & 215                                        & Wanderspalt    & PET-215.ms \\ \hline
%	\end{tabular}
%\end{center}

\chapter{Auswertung}
\section{Überprüfung der Vorhersagen der klassischen Erlärung}
\section{Überprüfung der Vorhersagen der quantenmeschanischen Erklärung}
\section{Bestimmung des Bohrschen Magnetons}
%\begin{center}
%\begin{figure}[h]
% \begin{tikzpicture}
% \begin{axis}[
%     legend pos=south west,
%     title={Messereignisse in Abhängigkeit der Detektorposition},
%     xlabel=Detektorposition in µm,
%     x tick label style={/pgf/number format/1000 sep=},
%     ylabel=Messereignisse,
%     y tick label style={/pgf/number format/1000 sep=},
%     width=0.9\textwidth,
%     height= 9cm,
%     xmin=1500,
%     xmax=2500,
%     grid=both,
%     ymin=0,
%     %ymax=0.0045,
%     tick align=outside,
%     tickpos=left,
%     minor x tick num=3,
%     minor y tick num=4,
%     minor grid style={dotted,thin}
% ]
% \addplot[red, only marks, mark=x, mark size=1pt, %error bars/.cd, y dir=both, y fixed relative=0.01, x dir=both, x fixed=0.05
% ]
% table[x index={0},y index={1}] {data/beamprofile2.bp};
% %\addlegendentry{Leistung der LED auf der Photoplatte}
% \end{axis}
% \end{tikzpicture}
% \captionof{figure}{Zahl der Messereignisse über der Detektorpositon mit leeren Probenhalter, ohne beamstop und mit Messing-Strahlabsorber.}
% \end{figure}
% \end{center}
\chapter{Fazit}

%ENDE INHALT
\cleardoublepage{}
% Eintrag fürs Inhaltsverzeichnis
\newpage
\begin{thebibliography}{100}
  \bibitem{Anleitung} \url{Versuchsanleitung}
\end{thebibliography}
\end{document}
